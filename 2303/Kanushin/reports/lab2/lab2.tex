\documentclass[14pt, a4paper]{extarticle}

\usepackage[utf8]{inputenc}
\usepackage[russian]{babel}

\usepackage{geometry}
\geometry{
	right=0.39in,
	left=1.18in,
	top=0.3in,
	bottom=0.49in
}

\usepackage{indentfirst}
\setlength\parindent{1.25cm}

\usepackage{titlesec}
\titlespacing*{\section}{0pt}{10pt}{0pt}

\usepackage{titlesec}
\titleformat{\section}{\normalfont\bfseries}{}{0pt}{}
\titleformat{\subsection}{\normalfont\itshape}{}{0pt}{}	
	
\usepackage{setspace}
\onehalfspacing

\usepackage{hyperref}

\usepackage{listings}
\usepackage{xcolor}
\lstset { %
	language=C++,
	backgroundcolor=\color{black!5}, % set backgroundcolor
	basicstyle=\footnotesize,% basic font setting
}

\def \deptName {МО ЭВМ}
\def \subjName {Системы реального времени на основе Linux}
\def \labNo {2}
\def \labName {Использование сервисов}

\def \groupNo {2303}
\def \studName {Канушин М.С.}
\def \proffName {Филатов А.Ю.}

\begin{document}
	
	\begin{titlepage}
	\begin{center}
		\textbf{МИНОБРНАУКИ РОССИИ \\
		САНКТ-ПЕТЕРБУРГСКИЙ ГОСУДАРСТВЕННЫЙ \\	
		ЭЛЕКТРОТЕХНИЧЕСКИЙ УНИВЕРСИТЕТ \\
		<<ЛЭТИ>> ИМ. В.И. УЛЬЯНОВА (ЛЕНИНА) \\
		Кафедра \deptName}
		
		\topskip0pt
		\vspace*{\fill}
			\bigskip\bigskip\bigskip\bigskip\bigskip
			\bigskip\bigskip\bigskip\bigskip\bigskip
			\textbf{ОТЧЕТ \\
			по лабораторной работе №\labNo \\
			по дисциплине <<\subjName>> \\
			Тема: \labName}
		\vspace*{\fill}
		
		\vspace*{\fill}
		\begin{tabular*}{\textwidth}{l @{\extracolsep{\fill}} r r}
			Студент гр. \groupNo & \noindent\rule{4cm}{0.4pt} & \studName \\
			Преподаватель        & \noindent\rule{4cm}{0.4pt} & \proffName \\
		\end{tabular*}
	
		\bigskip\bigskip\bigskip
		\bigskip\bigskip\bigskip
		
		Санкт-Петербург \\
		2018
	\end{center}
	\end{titlepage}
	\setcounter{page}{2}
	
	\section{Цель работы.}
	Разработать программу, моделирующую произвольную систему с использованием сервисов.

	\section{Описание системы.}
	Космический корабль состоит из модулей. Корабль может периодически получать повреждения, при этом ущерб наносится случайному его модулю. При сильном повреждении модуль начинает отправлять сообщения с просьбой о ремонте. Если hp модуля упало до 0, то он не подлежит восстановлению. Корабль также обладает сервисом ремонта, который получает имя модуля, производит ремонт и возвращает 0, если операции прошла успешно, 1 - если модуль не подлежит ремонту и -1 - если такого модуля на корабле нет. При полном отказе нескольких модулей (по умолчанию 3х) корабль отключается.
	
	Терминал при запуске ожидает запуска сервиса ремонта на корабле. При получении информации от модулей, терминал заносит информацию в локальное хранилище для последующей обработки пользователем.
	
	\section{Команды оператора}
	
	\begin{itemize}
		\item log - выводит последние полезные сообщения терминала.
		\item status - выводит информацию о состоянии поврежденных и не подлежащих восстановлению модулей из локального хранилища.
		\item repair <module\_name> - производит вызов сервиса и попытку ремонта.
	\end{itemize}
	
	\section{Спецификация программы}

	\begin{itemize}
		\item Топики
			\begin{itemize}
				\item damaged
				\item ruined
			\end{itemize}
		\item Сервисы
			\begin{itemize}
				\item repair\_service
			\end{itemize}
		\item Классы
			\begin{itemize}
				\item spaceship.Spaceship - космический корабль
				\item spaceship.Module - модуль космического корабля
				\item terminal.Terminal - терминал
			\end{itemize}
	\end{itemize}

	\section{Выводы.}
	В ходе выполнения данной лабораторной работы были изучена структура проекта в среде ROS и разработана программа, использующая технологию publisher-subscriber.
	
	\section{Код программы.}
	Код программы представлен в репозитории по адресу \url{https://github.com/Rextuz/ros_2017/tree/master/2303/Kanushin/lab2}
	
\end{document}