\documentclass[14pt, a4paper]{extarticle}

\usepackage[utf8]{inputenc}
\usepackage[russian]{babel}

\usepackage{geometry}
\geometry{
	right=0.39in,
	left=1.18in,
	top=0.3in,
	bottom=0.49in
}

\usepackage{indentfirst}
\setlength\parindent{1.25cm}

\usepackage{titlesec}
\titlespacing*{\section}{0pt}{10pt}{0pt}

\usepackage{titlesec}
\titleformat{\section}{\normalfont\bfseries}{}{0pt}{}
\titleformat{\subsection}{\normalfont\itshape}{}{0pt}{}	
	
\usepackage{setspace}
\onehalfspacing

\usepackage{listings}
\usepackage{xcolor}
\lstset { %
	language=C++,
	backgroundcolor=\color{black!5}, % set backgroundcolor
	basicstyle=\footnotesize,% basic font setting
}

\def \deptName {МО ЭВМ}
\def \subjName {Системы реального времени на основе Linux}
\def \labNo {1}
\def \labName {Простая программа-писатель}

\def \groupNo {2303}
\def \studName {Канушин М.С.}
\def \proffName {Филатов А.Ю.}

\begin{document}
	
	\begin{titlepage}
	\begin{center}
		\textbf{МИНОБРНАУКИ РОССИИ \\
		САНКТ-ПЕТЕРБУРГСКИЙ ГОСУДАРСТВЕННЫЙ \\	
		ЭЛЕКТРОТЕХНИЧЕСКИЙ УНИВЕРСИТЕТ \\
		<<ЛЭТИ>> ИМ. В.И. УЛЬЯНОВА (ЛЕНИНА) \\
		Кафедра \deptName}
		
		\topskip0pt
		\vspace*{\fill}
			\bigskip\bigskip\bigskip\bigskip\bigskip
			\bigskip\bigskip\bigskip\bigskip\bigskip
			\textbf{ОТЧЕТ \\
			по лабораторной работе №\labNo \\
			по дисциплине <<\subjName>> \\
			Тема: \labName}
		\vspace*{\fill}
		
		\vspace*{\fill}
		\begin{tabular*}{\textwidth}{l @{\extracolsep{\fill}} r r}
			Студент гр. \groupNo & \noindent\rule{4cm}{0.4pt} & \studName \\
			Преподаватель        & \noindent\rule{4cm}{0.4pt} & \proffName \\
		\end{tabular*}
	
		\bigskip\bigskip\bigskip
		\bigskip\bigskip\bigskip
		
		Санкт-Петербург \\
		2018
	\end{center}
	\end{titlepage}
	\setcounter{page}{2}
	
	\section{Цель работы.}
	Разработать программу, моделирующую робота и лабиринт. Робот должен управляться пользователем через технологию publisher-subscriber. Робот перемещается только по свободным клеткам, перемещение через препятствия(стены) не допускается.

	\section{Основные теоретические положения.}
	У объекта класса NodeHandle
	
	\begin{lstlisting}
	ros::NodeHandle <node_handle_name>;
	\end{lstlisting}
	
	есть метод, реализующий механизм отправки сообщений в топик с именем <topic\_name>.
	
	Это делается при помощи команды
	
	\begin{lstlisting}
	ros::Publisher <publisher_name> = <node_handle_name>.advertise<msg_type>("<topic_name>",<size>);
	\end{lstlisting}
	
	где <topic\_name> это имя топика, через который будут общаться publisher и subscriber; а <size> - размер буфера сообщений (а треугольные скобочки после advertize - это конкретизация шаблонной функции).
	
	В данном случае имя топика можно получить, узнав, на какой топик подписан turtlesim\_node. Тип сообщения можно узнать, выведя информацию о топике.
	
	Информацию о том, из каких полей состоит сообщение можно узнать командой
	
	\begin{lstlisting}
	rosmsg show <msg_type>	
	\end{lstlisting}
	
	В тексте программы необходимо создать объект класса <msg\_type> и наполнить его содержимым. Информативными являются поля msg.linear.x и msg.angular.z. Остальные поля сообщения не учитываются при обработке.
	
	После того, как сообщение сформировано, его можно отправить в топик командой 	
	\begin{lstlisting}
		<publisher_name>.publish (msg)	
	\end{lstlisting}
		
	В разделе может быть приведено описание исследуемых физических явлений (с иллюстрациями), основные теоретические положения (в том числе – математический аппарат, описывающий исследуемые явления), схемы измерений, сведения об используемом при проведении работы лабораторном оборудовании.

	\section{Разработка программы.}
	Для отправки сообщений роботу будет использоваться единственный топик - \textit{controller}, отвечающий за команды перемещения.
	
	\subsection{Узлы.}
	\begin{itemize}
	\item controller - узел, считывающий команды пользователя и отправляющий их в топик \textit{controller}
	\item listener - узел, отвечающий за прием сообщений из топика \textit{controller} и производящий перемещение робота в случае, если это возможно.
	\end{itemize}

	\subsection{Классы.}
	\begin{itemize}
		\item KeyPublisher - класс, реализующий функциональность узла \textit{controller}
		\item LevelMap - класс, хранящий карту лабиринта и местоположение робота на ней.
		\item Robot - класс, реализующий функциональность робота. Робот определят свое местоположение на карте, считывает из топика команды и выполняет их в случае, если это возможно.
	\end{itemize}

	\section{Выводы.}
	В ходе выполнения данной лабораторной работы были изучена структура проекта в среде ROS и разработана программа, использующая технологию publisher-subscriber.
	
	\section{Код программы.}
	\begin{lstlisting}
#!/usr/bin/env python

import rospy
from pynput.keyboard import Key, Listener, KeyCode
from std_msgs.msg import String


class KeyPublisher:
	KEY_Q = KeyCode(char='q')
	KEY_W = KeyCode(char='w')
	KEY_A = KeyCode(char='a')
	KEY_S = KeyCode(char='s')
	KEY_D = KeyCode(char='d')
	KEY_P = KeyCode(char='p')
	keys = (KEY_Q, KEY_W, KEY_A, KEY_S, KEY_D, KEY_P, Key.esc)

	def __init__(self, topic):
		self.topic = topic
		self.pub = rospy.Publisher(topic, String, queue_size=10)
		rospy.init_node('key_talker', anonymous=True)

	def start(self):
		def __on_press(key):
			if key in self.keys:
				rospy.loginfo('got key %s', key)
				if key == self.KEY_W:
					self.pub.publish('up')
				if key == self.KEY_S:
					self.pub.publish('down')
				if key == self.KEY_A:
					self.pub.publish('left')
				if key == self.KEY_D:
					self.pub.publish('right')
				if key == self.KEY_P:
					self.pub.publish('print')

		def __on_release(key):
			if key in (Key.esc, self.KEY_Q):
				return False

		rospy.loginfo('Controller with topic `%s` launched',
			  self.topic)
		rospy.loginfo('w, a, s, d to move')
		rospy.loginfo('esc, q to exit')
		rospy.loginfo('p to print level map')

		with Listener(on_press=__on_press,
			  on_release=__on_release) as listener:
			listener.join()


if __name__ == '__main__':
	try:
		talker = KeyPublisher('controller')
		talker.start()
	except rospy.ROSInterruptException:
		pass
	\end{lstlisting}
	
	\begin{lstlisting}
#!/usr/bin/env python

import rospy
import rospkg
from std_msgs.msg import String


class LevelMap:
	def __init__(self, map_file):
		self.map = []
		with open(map_file) as f:
			for line in f:
				tmp = []
				for char in line:
					tmp.append(char)
				self.map.append(tmp)

	def free(self, x, y):
		return self.map[x][y] == '0'

	def print_map(self, x, y):
		self.map[x][y] = '*'
		res = '\nLevel map:\n'
		for line in self.map:
			res += ''.join(line)
		res += '* - robot\n'
		res += '1 - wall\n'
		res += '0 - empty space\n'
		rospy.loginfo(res)
		self.map[x][y] = '0'


class Robot:
	def __init__(self, name, x=0, y=0):
		self.name = name
		self.x = x
		self.y = y
		self.map = None

	def load_map(self, local_map):
		self.map = local_map

	def move(self, dx, dy):
		new_x = self.x + dx
		new_y = self.y + dy
		if self.map.free(new_x, new_y):
			rospy.loginfo('I am %s, I am moving to (%d,%d)', self.name,
				  new_x, new_y)
			self.x = new_x
			self.y = new_y
		else:
			rospy.loginfo('I am %s, cannot move there', self.name)

	def show_location(self):
		self.map.print_map(self.x, self.y)

	def subscribe(self, topic):
		def callback(data):
			rospy.loginfo('I am %s, my id is %s, I received %s',
				  self.name,
				  rospy.get_caller_id(),
				  data.data)
			if data.data == 'up':
				self.move(-1, 0)

			if data.data == 'left':
				self.move(0, -1)

			if data.data == 'right':
				self.move(0, 1)

			if data.data == 'down':
				self.move(1, 0)

			if data.data == 'print':
				self.show_location()

		rospy.init_node('listener', anonymous=True)
		rospy.Subscriber(topic, String, callback)
		rospy.spin()


if __name__ == '__main__':
	rospack = rospkg.RosPack()
	res_path = rospack.get_path('lab_1') + '/res/'
	level = LevelMap('{}map.txt'.format(res_path))

	andy = Robot('Andy', 4, 3)
	andy.load_map(level)
	andy.subscribe('controller')
	\end{lstlisting}
	
\end{document}